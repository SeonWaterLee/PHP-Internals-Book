\section{资源类型:zend\_resource}
\label{sec:zend_resource}

实际上尽管PHP已经可以丢弃“资源”类型,因为自定义对象存储允许构建任何抽象类型的数据,在当前PHP版本中仍然存在资源类型,你还是可能需要处理它。

如果你需要创建资源,我们希望您不要这样做,而是使用对象和自定义对象存储处理。对象(Objects)是PHP的一种类型,它可以嵌入任何类型的任何内容。然而因为历史原因,PHP仍然要知道这种特殊类型的“资源” ,并且在其核心代码或者一些扩展中仍然在使用它。我们一起了解一下这种类型。但是要注意,它确实很神秘并且历史悠久,所以不要对它的设计感到惊讶,特别是在阅读有关它的源码时。

\subsection{什么是“资源”类型?}

你很容易就知道了。我们正在讨论这个:

\begin{lstlisting}[language=php]
$fp = fopen('/proc/cpuinfo', 'r');
var_dump($fp); /* resource(2) of type (stream) */
\end{lstlisting}

在内部,资源被绑定到\code{zend\_resource}结构类型:

\begin{lstlisting}[language=c]
struct _zend_resource {
        zend_refcounted_h gc;
        int               handle;
        int               type;
        void             *ptr;
};
\end{lstlisting}

我们发现了通用的\code{zend\_refcounted\_h}头,这意味这资源是可引用的。\code{handle}是整数类型,它在引擎内部被用来将资源定位到内部资源表中。它被用作此类表的键。

\code{type}被用来将相同类型的资源重组在一起。这是关于资源被销毁的方式以及如何从它们的handle中取回。

最后,在\code{zend\_resource}的\code{ptr}字段是你的抽象数据。请记住,资源是关于存储一个抽象数据的,它不能适合PHP原生的任何数据类型(但是对象可以,就像我们之前说的那样)。

\subsection{资源类型和资源销毁}

资源必须注册析构函数。当用户在PHP用户层使用资源时,当它们不再使用过这些资源时,它们也不会费心清理它们。例如:看见\code{fopen()}调用但是没有看到对应的\code{fclose()}调用是不稀罕的。使用C语言,这充其量是个坏主意,至多也就是一场灾难。但是使用像PHP这样的高级语言,事情就容易多了。

作为内部开发人员,你必须面对这样一个事实:用户将会创建许多你运行它们使用的资源,而且没有正确的清理它们并释放内存/系统资源。因此你必须一个析构函数,该析构函数将在引擎即将销毁该类型的资源时被调用。

析构函数是按照类型分组的,资源也是这样。你不能请求“dtabase”类型资源的析构函数来来处理"file"类型的资源。

还有两种资源,它们的生命周期是不同的。

\begin{itemize}
        \item  经典资源,最常用的一种,不会跨多个请求存留,它们的析构函数会在请求关闭时被调用。
        \item 持久性资源将持续存在于多个请求中,并且只有在PHP进程终止时才会被销毁。
\end{itemize}


\alertinfo{你可能会对\href{http://www.phpinternalsbook.com/php7/extensions_design/php_lifecycle.html}{PHP生命周期}这一章感兴趣,该章节展示了在PHP的流程生命周期内发生的不同步骤。此外,\href{http://www.phpinternalsbook.com/php7/memory_management/zend_memory_manager.html}{Zend Memory Manager}章节可能有助于理解持久性和请求绑定内存分配的概念。}

\subsection{使用资源}

与资源相关的API可以在\href{https://github.com/php/php-src/blob/3704947696fe0ee93e025fa85621d297ac7a1e4d/Zend/zend_list.c}{zend/zend\_list.c}中查到。您可能会发现其中有一些不一致的地方,比如讨论“资源”的“列表”时。

\subsubsection{创建资源}

要创建资源,首先必须要为它注册析构函数,并使用\code{zend\_register\_list\_destructors\_ex()}将其关联到一个资源类型名称。此函数将返回一个表示你注册的资源类型的整数。该调用将返回一个整数,该整数表示您注册的资源类型。你必须记住该整数,因为你稍后需要从它那里获取资源。

然后,可以使用\code{zend\_register\_resource()}注册一个新的资源。它将会返回一个\code{zend\_resource}。让我们一起看一个简单的例子:

\begin{lstlisting}[language=c]
        #include <stdio.h>

        int res_num;
        FILE *fp;
        zend_resource *my_res;
        zval my_val;
        
        static void my_res_dtor(zend_resource *rsrc)
        {
            fclose((FILE *)rsrc->ptr);
        }
        
        /* module_number should be your PHP extension number here */
        res_num = zend_register_list_destructors_ex(my_res_dtor, NULL, "my_res", module_number);
        fp      = fopen("/proc/cpuinfo", "r");
        my_res  = zend_register_resource((void *)fp, res_num);
        
        ZVAL_RES(&my_val, my_res); 
\end{lstlisting}

在上面的代码中,我们使用libc的\code{fopen()}打开了一个文件,并将返回的指针存储到一个资源中。在此之前,我们注册了一个析构函数,当他被调用时它会调用libc's的\code{fclose()}。然后,我们将资源注册到引擎上,并将资源传递到一个\code{zval}容器中,它可以在用户层获取到。

\alertinfo{Zvals章节在\hyperref[sec:zvals]{这里}}

资源类型必须要记住。在这里我们注册了一个类型为“my\_res“的资源。这是类型的名称。引擎实际上并不关注类型名称,而是类型标识符,即\code{zend\_register\_list\_destructors\_ex()}返回的整数。你应该在某处记下它,就像我们使用\code{res\_num}变量那样。

\subsubsection{获取资源}
现在我们已经注册了一个资源并将其放入\code{zval}中作为实例,我们应该学习如何从用户层获取该资源。记住,该资源存储在\code{zval}中。资源结构中存储了资源类型编号(在type字段中)。
因此,从用户层面获取资源,我们必须从\code{zval}中提取\code{zend\_resource},并调用\code{zend\_fetch\_resource()}来获取我们的\code{FILE * }指针:

\begin{lstlisting}[language=c]
        /* ... later on ... */

        zval *user_zval = /* fetch zval from userland, assume type IS_RESOURCE */

        ZEND_ASSERT(Z_TYPE_P(user_zval) == IS_RESOURCE); /* just a check to be sure */

        fp = (FILE *)zend_fetch_resource(Z_RESVAL_P(user_zval), "my_res", res_num);
\end{lstlisting}        

如前所述: 从用户层面获取一个\code{zval}(类型为IS\_RESOURCE),并痛殴调用\code{,并通过调用zend\_fetch\_resource()}获取资源指针。

该函数将检查资源的类型是否与作为第三个参数传递的类型相同(这里是\code{res\_num})。如果相同,它将返回你需要的\code{void *}资源指针,我们就完成了。如果不同,那么它会抛出一个类似于“supplied resource is not a valid {type name} resource”的警告。这种情况就会发生,例如,你期望得到一个“my\_res"类型的资源,但是你得到了一个资料类型为“gzip”的\code{zval},例如PHP函数 \code{gzopen()}返回的那样,那么就会发生这种情况。

资源类型只是引擎将不同类型的资源(类型为“file”、“gzip”甚至“mysql connection”)混合到同一个资源表中的一种方法。资源类型有名称,因此可以在错误消息或调试语句中使用这些名称(比如\code{var\_dump(\$my\_resource)}),并且它们还被表示为一个整数,用于在内部从资源类型中取回资源指针,并注册一个析构函数。

\alertinfo{如您所见,如果我们使用对象,它们自身就表示类型,并且不需要从它的标识符获取资源来验证它的类型。对象是自描述类型。但是对于当前的PHP版本,资源仍然是有效的数据类型}

\subsection{资源引用计数}

与许多其他类型一样,\code{zend\_resource}也是引用计数的。我们可以看到它的\code{zend\_refcounted\_h}头。如果你需要的话(一般来说你不应该真的需要它),可以使用引用计数的API:

\begin{itemize}
        \item \code{zend\_list\_delete(zend\_resource *res)} 递减引用计数,如果为零可销毁资源
        \item \code{zend\_list\_free(zend\_resource *res)} 检查refcount是否为零,如果是则销毁资源。
        \item \code{zend\_list\_close(zend\_resource *res)} 在任何条件下调用资源析构函数
\end{itemize}


\subsection{持久化资源}

持久化资源在请求结束时不会被销毁。经典用例是持久化数据库连接。这些连接是在请求之间循环使用的。传统上,您不应该使用持久化资源,因为每个请求都是不同的。在这样做之前,重用相同的资源确实需要深思熟虑。

要注册持久化资源,请使用持久析构函数而不是经典析构函数。这是在\code{zend\_register\_list\_destructors\_ex()}调用时完成的,该API类似于:

\begin{lstlisting}[language=c]
        zend_register_list_destructors_ex(rsrc_dtor_func_t destructor, rsrc_dtor_func_t persistent_destructor,
        const char *type_name, int module_number);     
\end{lstlisting} 



